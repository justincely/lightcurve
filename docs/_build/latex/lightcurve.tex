% Generated by Sphinx.
\def\sphinxdocclass{report}
\documentclass[letterpaper,10pt,english]{sphinxmanual}
\usepackage[utf8]{inputenc}
\DeclareUnicodeCharacter{00A0}{\nobreakspace}
\usepackage[T1]{fontenc}
\usepackage{babel}
\usepackage{times}
\usepackage[Bjarne]{fncychap}
\usepackage{longtable}
\usepackage{sphinx}
\usepackage{multirow}


\title{lightcurve Documentation}
\date{January 15, 2014}
\release{0.1.1}
\author{Justin Ely}
\newcommand{\sphinxlogo}{}
\renewcommand{\releasename}{Release}
\makeindex

\makeatletter
\def\PYG@reset{\let\PYG@it=\relax \let\PYG@bf=\relax%
    \let\PYG@ul=\relax \let\PYG@tc=\relax%
    \let\PYG@bc=\relax \let\PYG@ff=\relax}
\def\PYG@tok#1{\csname PYG@tok@#1\endcsname}
\def\PYG@toks#1+{\ifx\relax#1\empty\else%
    \PYG@tok{#1}\expandafter\PYG@toks\fi}
\def\PYG@do#1{\PYG@bc{\PYG@tc{\PYG@ul{%
    \PYG@it{\PYG@bf{\PYG@ff{#1}}}}}}}
\def\PYG#1#2{\PYG@reset\PYG@toks#1+\relax+\PYG@do{#2}}

\def\PYG@tok@gd{\def\PYG@tc##1{\textcolor[rgb]{0.63,0.00,0.00}{##1}}}
\def\PYG@tok@gu{\let\PYG@bf=\textbf\def\PYG@tc##1{\textcolor[rgb]{0.50,0.00,0.50}{##1}}}
\def\PYG@tok@gt{\def\PYG@tc##1{\textcolor[rgb]{0.00,0.25,0.82}{##1}}}
\def\PYG@tok@gs{\let\PYG@bf=\textbf}
\def\PYG@tok@gr{\def\PYG@tc##1{\textcolor[rgb]{1.00,0.00,0.00}{##1}}}
\def\PYG@tok@cm{\let\PYG@it=\textit\def\PYG@tc##1{\textcolor[rgb]{0.25,0.50,0.56}{##1}}}
\def\PYG@tok@vg{\def\PYG@tc##1{\textcolor[rgb]{0.73,0.38,0.84}{##1}}}
\def\PYG@tok@m{\def\PYG@tc##1{\textcolor[rgb]{0.13,0.50,0.31}{##1}}}
\def\PYG@tok@mh{\def\PYG@tc##1{\textcolor[rgb]{0.13,0.50,0.31}{##1}}}
\def\PYG@tok@cs{\def\PYG@tc##1{\textcolor[rgb]{0.25,0.50,0.56}{##1}}\def\PYG@bc##1{\colorbox[rgb]{1.00,0.94,0.94}{##1}}}
\def\PYG@tok@ge{\let\PYG@it=\textit}
\def\PYG@tok@vc{\def\PYG@tc##1{\textcolor[rgb]{0.73,0.38,0.84}{##1}}}
\def\PYG@tok@il{\def\PYG@tc##1{\textcolor[rgb]{0.13,0.50,0.31}{##1}}}
\def\PYG@tok@go{\def\PYG@tc##1{\textcolor[rgb]{0.19,0.19,0.19}{##1}}}
\def\PYG@tok@cp{\def\PYG@tc##1{\textcolor[rgb]{0.00,0.44,0.13}{##1}}}
\def\PYG@tok@gi{\def\PYG@tc##1{\textcolor[rgb]{0.00,0.63,0.00}{##1}}}
\def\PYG@tok@gh{\let\PYG@bf=\textbf\def\PYG@tc##1{\textcolor[rgb]{0.00,0.00,0.50}{##1}}}
\def\PYG@tok@ni{\let\PYG@bf=\textbf\def\PYG@tc##1{\textcolor[rgb]{0.84,0.33,0.22}{##1}}}
\def\PYG@tok@nl{\let\PYG@bf=\textbf\def\PYG@tc##1{\textcolor[rgb]{0.00,0.13,0.44}{##1}}}
\def\PYG@tok@nn{\let\PYG@bf=\textbf\def\PYG@tc##1{\textcolor[rgb]{0.05,0.52,0.71}{##1}}}
\def\PYG@tok@no{\def\PYG@tc##1{\textcolor[rgb]{0.38,0.68,0.84}{##1}}}
\def\PYG@tok@na{\def\PYG@tc##1{\textcolor[rgb]{0.25,0.44,0.63}{##1}}}
\def\PYG@tok@nb{\def\PYG@tc##1{\textcolor[rgb]{0.00,0.44,0.13}{##1}}}
\def\PYG@tok@nc{\let\PYG@bf=\textbf\def\PYG@tc##1{\textcolor[rgb]{0.05,0.52,0.71}{##1}}}
\def\PYG@tok@nd{\let\PYG@bf=\textbf\def\PYG@tc##1{\textcolor[rgb]{0.33,0.33,0.33}{##1}}}
\def\PYG@tok@ne{\def\PYG@tc##1{\textcolor[rgb]{0.00,0.44,0.13}{##1}}}
\def\PYG@tok@nf{\def\PYG@tc##1{\textcolor[rgb]{0.02,0.16,0.49}{##1}}}
\def\PYG@tok@si{\let\PYG@it=\textit\def\PYG@tc##1{\textcolor[rgb]{0.44,0.63,0.82}{##1}}}
\def\PYG@tok@s2{\def\PYG@tc##1{\textcolor[rgb]{0.25,0.44,0.63}{##1}}}
\def\PYG@tok@vi{\def\PYG@tc##1{\textcolor[rgb]{0.73,0.38,0.84}{##1}}}
\def\PYG@tok@nt{\let\PYG@bf=\textbf\def\PYG@tc##1{\textcolor[rgb]{0.02,0.16,0.45}{##1}}}
\def\PYG@tok@nv{\def\PYG@tc##1{\textcolor[rgb]{0.73,0.38,0.84}{##1}}}
\def\PYG@tok@s1{\def\PYG@tc##1{\textcolor[rgb]{0.25,0.44,0.63}{##1}}}
\def\PYG@tok@gp{\let\PYG@bf=\textbf\def\PYG@tc##1{\textcolor[rgb]{0.78,0.36,0.04}{##1}}}
\def\PYG@tok@sh{\def\PYG@tc##1{\textcolor[rgb]{0.25,0.44,0.63}{##1}}}
\def\PYG@tok@ow{\let\PYG@bf=\textbf\def\PYG@tc##1{\textcolor[rgb]{0.00,0.44,0.13}{##1}}}
\def\PYG@tok@sx{\def\PYG@tc##1{\textcolor[rgb]{0.78,0.36,0.04}{##1}}}
\def\PYG@tok@bp{\def\PYG@tc##1{\textcolor[rgb]{0.00,0.44,0.13}{##1}}}
\def\PYG@tok@c1{\let\PYG@it=\textit\def\PYG@tc##1{\textcolor[rgb]{0.25,0.50,0.56}{##1}}}
\def\PYG@tok@kc{\let\PYG@bf=\textbf\def\PYG@tc##1{\textcolor[rgb]{0.00,0.44,0.13}{##1}}}
\def\PYG@tok@c{\let\PYG@it=\textit\def\PYG@tc##1{\textcolor[rgb]{0.25,0.50,0.56}{##1}}}
\def\PYG@tok@mf{\def\PYG@tc##1{\textcolor[rgb]{0.13,0.50,0.31}{##1}}}
\def\PYG@tok@err{\def\PYG@bc##1{\fcolorbox[rgb]{1.00,0.00,0.00}{1,1,1}{##1}}}
\def\PYG@tok@kd{\let\PYG@bf=\textbf\def\PYG@tc##1{\textcolor[rgb]{0.00,0.44,0.13}{##1}}}
\def\PYG@tok@ss{\def\PYG@tc##1{\textcolor[rgb]{0.32,0.47,0.09}{##1}}}
\def\PYG@tok@sr{\def\PYG@tc##1{\textcolor[rgb]{0.14,0.33,0.53}{##1}}}
\def\PYG@tok@mo{\def\PYG@tc##1{\textcolor[rgb]{0.13,0.50,0.31}{##1}}}
\def\PYG@tok@mi{\def\PYG@tc##1{\textcolor[rgb]{0.13,0.50,0.31}{##1}}}
\def\PYG@tok@kn{\let\PYG@bf=\textbf\def\PYG@tc##1{\textcolor[rgb]{0.00,0.44,0.13}{##1}}}
\def\PYG@tok@o{\def\PYG@tc##1{\textcolor[rgb]{0.40,0.40,0.40}{##1}}}
\def\PYG@tok@kr{\let\PYG@bf=\textbf\def\PYG@tc##1{\textcolor[rgb]{0.00,0.44,0.13}{##1}}}
\def\PYG@tok@s{\def\PYG@tc##1{\textcolor[rgb]{0.25,0.44,0.63}{##1}}}
\def\PYG@tok@kp{\def\PYG@tc##1{\textcolor[rgb]{0.00,0.44,0.13}{##1}}}
\def\PYG@tok@w{\def\PYG@tc##1{\textcolor[rgb]{0.73,0.73,0.73}{##1}}}
\def\PYG@tok@kt{\def\PYG@tc##1{\textcolor[rgb]{0.56,0.13,0.00}{##1}}}
\def\PYG@tok@sc{\def\PYG@tc##1{\textcolor[rgb]{0.25,0.44,0.63}{##1}}}
\def\PYG@tok@sb{\def\PYG@tc##1{\textcolor[rgb]{0.25,0.44,0.63}{##1}}}
\def\PYG@tok@k{\let\PYG@bf=\textbf\def\PYG@tc##1{\textcolor[rgb]{0.00,0.44,0.13}{##1}}}
\def\PYG@tok@se{\let\PYG@bf=\textbf\def\PYG@tc##1{\textcolor[rgb]{0.25,0.44,0.63}{##1}}}
\def\PYG@tok@sd{\let\PYG@it=\textit\def\PYG@tc##1{\textcolor[rgb]{0.25,0.44,0.63}{##1}}}

\def\PYGZbs{\char`\\}
\def\PYGZus{\char`\_}
\def\PYGZob{\char`\{}
\def\PYGZcb{\char`\}}
\def\PYGZca{\char`\^}
\def\PYGZsh{\char`\#}
\def\PYGZpc{\char`\%}
\def\PYGZdl{\char`\$}
\def\PYGZti{\char`\~}
% for compatibility with earlier versions
\def\PYGZat{@}
\def\PYGZlb{[}
\def\PYGZrb{]}
\makeatother

\begin{document}

\maketitle
\tableofcontents
\phantomsection\label{index::doc}


Contents:
\phantomsection\label{index:module-lightcurve.lightcurve}\index{lightcurve.lightcurve (module)}
Contains the basic LightCurve object which is inherited by the subclasses.
\index{LightCurve (class in lightcurve.lightcurve)}

\begin{fulllineitems}
\phantomsection\label{index:lightcurve.lightcurve.LightCurve}\pysigline{\strong{class }\code{lightcurve.lightcurve.}\bfcode{LightCurve}}~\begin{quote}\begin{description}
\item[{Returns }] \leavevmode
\textbf{lightcurve object} :

\end{description}\end{quote}
\paragraph{Methods}
\index{counts (lightcurve.lightcurve.LightCurve attribute)}

\begin{fulllineitems}
\phantomsection\label{index:lightcurve.lightcurve.LightCurve.counts}\pysigline{\bfcode{counts}}
Calculate counts array

Counts are calculated as the gross - background
\begin{quote}\begin{description}
\item[{Returns }] \leavevmode
\textbf{counts} : np.ndarray
\begin{quote}

Array containing the calculated counts
\end{quote}

\end{description}\end{quote}

\end{fulllineitems}

\index{error (lightcurve.lightcurve.LightCurve attribute)}

\begin{fulllineitems}
\phantomsection\label{index:lightcurve.lightcurve.LightCurve.error}\pysigline{\bfcode{error}}
Calculate error array

Error is calculated as sqrt( gross + background )
\begin{quote}\begin{description}
\item[{Returns }] \leavevmode
\textbf{error} : np.ndarray
\begin{quote}

Array containing the calculated error
\end{quote}

\end{description}\end{quote}

\end{fulllineitems}

\index{flux\_error (lightcurve.lightcurve.LightCurve attribute)}

\begin{fulllineitems}
\phantomsection\label{index:lightcurve.lightcurve.LightCurve.flux_error}\pysigline{\bfcode{flux\_error}}
Estimate the error in the flux array

Flux error is currently estimated as having the same magnitude
relative to flux as the error has to counts.  Very simple approximation
for now
\begin{quote}\begin{description}
\item[{Returns }] \leavevmode
\textbf{flux\_error} : np.ndarray
\begin{quote}

Array containing the calculated error in flux
\end{quote}

\end{description}\end{quote}

\end{fulllineitems}

\index{net (lightcurve.lightcurve.LightCurve attribute)}

\begin{fulllineitems}
\phantomsection\label{index:lightcurve.lightcurve.LightCurve.net}\pysigline{\bfcode{net}}
Calculate net array

Net is calculated as counts / time
\begin{quote}\begin{description}
\item[{Returns }] \leavevmode
\textbf{net} : np.ndarray
\begin{quote}

Array containing calculated net
\end{quote}

\end{description}\end{quote}

\end{fulllineitems}

\index{signal\_to\_noise (lightcurve.lightcurve.LightCurve attribute)}

\begin{fulllineitems}
\phantomsection\label{index:lightcurve.lightcurve.LightCurve.signal_to_noise}\pysigline{\bfcode{signal\_to\_noise}}
Quick signal to noise estimate

S/N is calculated as the ratio of gross to error
\begin{quote}\begin{description}
\item[{Returns }] \leavevmode
\textbf{sn\_ratio} : np.ndarray
\begin{quote}

Array containing calculated signal to noise ratio
\end{quote}

\end{description}\end{quote}

\end{fulllineitems}

\index{write() (lightcurve.lightcurve.LightCurve method)}

\begin{fulllineitems}
\phantomsection\label{index:lightcurve.lightcurve.LightCurve.write}\pysiglinewithargsret{\bfcode{write}}{\emph{outname=None}, \emph{clobber=False}}{}
Write lightcurve out to FITS file
\begin{quote}\begin{description}
\item[{Parameters }] \leavevmode
\textbf{outname} : bool or str
\begin{quote}

Either True/False, or output name
\end{quote}

\textbf{clobber} : bool
\begin{quote}

Allow overwriting of existing file with same name
\end{quote}

\end{description}\end{quote}

\end{fulllineitems}


\end{fulllineitems}

\phantomsection\label{index:module-lightcurve.cos}\index{lightcurve.cos (module)}
Holder of Cosmic Origins Spectrograph (COS) classes and utilities
\index{extract\_index() (in module lightcurve.cos)}

\begin{fulllineitems}
\phantomsection\label{index:lightcurve.cos.extract_index}\pysiglinewithargsret{\code{lightcurve.cos.}\bfcode{extract\_index}}{\emph{hdu}, \emph{x\_start}, \emph{x\_end}, \emph{y\_start}, \emph{y\_end}, \emph{w\_start}, \emph{w\_end}, \emph{sdqflags=0}}{}
Extract event indeces from given HDU using input parameters.

Wavelength regions containing geocoronal airglow emission will be excluded
automatically.  These regions include Lyman Alpha between 1214 and 1217
Angstroms and Oxygen I from 1300 to 1308.
\begin{quote}\begin{description}
\item[{Parameters }] \leavevmode
\textbf{hdu} : HDUlist
\begin{quote}

Header Data Unit from COS corrtag file
\end{quote}

\textbf{x\_start} : float
\begin{quote}

Lower bound of events in pixel space
\end{quote}

\textbf{x\_end} : float
\begin{quote}

Upper bound of events in pixel space
\end{quote}

\textbf{y\_start} : float
\begin{quote}

Lower bound of events in pixel space
\end{quote}

\textbf{y\_end} : float
\begin{quote}

Upper bound of events in pixel space
\end{quote}

\textbf{w\_start} : float
\begin{quote}

Lower bound of events in wavelength space
\end{quote}

\textbf{w\_end} : float
\begin{quote}

Upper bound of events in wavelength space
\end{quote}

\textbf{sdqflags} : int
\begin{quote}

Bitwise DQ value of bad events
\end{quote}

\item[{Returns }] \leavevmode
\textbf{data\_index} : np.ndarray
\begin{quote}

Indeces of desired events
\end{quote}

\end{description}\end{quote}

\end{fulllineitems}

\begin{quote}\begin{description}
\item[{class}] \leavevmode
`CosCurve'

\end{description}\end{quote}


\bigskip\hrule{}\bigskip

\index{CosCurve (class in lightcurve.cos)}

\begin{fulllineitems}
\phantomsection\label{index:lightcurve.cos.CosCurve}\pysiglinewithargsret{\strong{class }\code{lightcurve.cos.}\bfcode{CosCurve}}{\emph{**kwargs}}{}
Bases: {\hyperref[index:lightcurve.lightcurve.LightCurve]{\code{lightcurve.lightcurve.LightCurve}}}

Subclass for COS specific routines and abilities
\paragraph{Methods}
\index{counts (lightcurve.cos.CosCurve attribute)}

\begin{fulllineitems}
\phantomsection\label{index:lightcurve.cos.CosCurve.counts}\pysigline{\bfcode{counts}}
Calculate counts array

Counts are calculated as the gross - background
\begin{quote}\begin{description}
\item[{Returns }] \leavevmode
\textbf{counts} : np.ndarray
\begin{quote}

Array containing the calculated counts
\end{quote}

\end{description}\end{quote}

\end{fulllineitems}

\index{error (lightcurve.cos.CosCurve attribute)}

\begin{fulllineitems}
\phantomsection\label{index:lightcurve.cos.CosCurve.error}\pysigline{\bfcode{error}}
Calculate error array

Error is calculated as sqrt( gross + background )
\begin{quote}\begin{description}
\item[{Returns }] \leavevmode
\textbf{error} : np.ndarray
\begin{quote}

Array containing the calculated error
\end{quote}

\end{description}\end{quote}

\end{fulllineitems}

\index{extract() (lightcurve.cos.CosCurve method)}

\begin{fulllineitems}
\phantomsection\label{index:lightcurve.cos.CosCurve.extract}\pysiglinewithargsret{\bfcode{extract}}{\emph{step=1}, \emph{xlim=(0}, \emph{16384)}, \emph{wlim=(2}, \emph{10000)}, \emph{ylim=None}}{}
Loop over HDUs and extract the lightcurve

\end{fulllineitems}

\index{flux\_error (lightcurve.cos.CosCurve attribute)}

\begin{fulllineitems}
\phantomsection\label{index:lightcurve.cos.CosCurve.flux_error}\pysigline{\bfcode{flux\_error}}
Estimate the error in the flux array

Flux error is currently estimated as having the same magnitude
relative to flux as the error has to counts.  Very simple approximation
for now
\begin{quote}\begin{description}
\item[{Returns }] \leavevmode
\textbf{flux\_error} : np.ndarray
\begin{quote}

Array containing the calculated error in flux
\end{quote}

\end{description}\end{quote}

\end{fulllineitems}

\index{net (lightcurve.cos.CosCurve attribute)}

\begin{fulllineitems}
\phantomsection\label{index:lightcurve.cos.CosCurve.net}\pysigline{\bfcode{net}}
Calculate net array

Net is calculated as counts / time
\begin{quote}\begin{description}
\item[{Returns }] \leavevmode
\textbf{net} : np.ndarray
\begin{quote}

Array containing calculated net
\end{quote}

\end{description}\end{quote}

\end{fulllineitems}

\index{read\_cos() (lightcurve.cos.CosCurve method)}

\begin{fulllineitems}
\phantomsection\label{index:lightcurve.cos.CosCurve.read_cos}\pysiglinewithargsret{\bfcode{read\_cos}}{\emph{filename}}{}
Extract light curve from COS data

\end{fulllineitems}

\index{signal\_to\_noise (lightcurve.cos.CosCurve attribute)}

\begin{fulllineitems}
\phantomsection\label{index:lightcurve.cos.CosCurve.signal_to_noise}\pysigline{\bfcode{signal\_to\_noise}}
Quick signal to noise estimate

S/N is calculated as the ratio of gross to error
\begin{quote}\begin{description}
\item[{Returns }] \leavevmode
\textbf{sn\_ratio} : np.ndarray
\begin{quote}

Array containing calculated signal to noise ratio
\end{quote}

\end{description}\end{quote}

\end{fulllineitems}

\index{write() (lightcurve.cos.CosCurve method)}

\begin{fulllineitems}
\phantomsection\label{index:lightcurve.cos.CosCurve.write}\pysiglinewithargsret{\bfcode{write}}{\emph{outname=None}, \emph{clobber=False}}{}
Write lightcurve out to FITS file
\begin{quote}\begin{description}
\item[{Parameters }] \leavevmode
\textbf{outname} : bool or str
\begin{quote}

Either True/False, or output name
\end{quote}

\textbf{clobber} : bool
\begin{quote}

Allow overwriting of existing file with same name
\end{quote}

\end{description}\end{quote}

\end{fulllineitems}


\end{fulllineitems}

\phantomsection\label{index:module-lightcurve.io}\index{lightcurve.io (module)}
I/O operations to get data into a LightCurve
\index{open() (in module lightcurve.io)}

\begin{fulllineitems}
\phantomsection\label{index:lightcurve.io.open}\pysiglinewithargsret{\code{lightcurve.io.}\bfcode{open}}{\emph{**kwargs}}{}
Open file into lightcurve

filename must be supplied in kwargs
\begin{quote}\begin{description}
\item[{Parameters }] \leavevmode
\textbf{**kwargs :} :
\begin{quote}

Additional arguements to be passed to lightcurve instantiations
\end{quote}

\item[{Returns }] \leavevmode
\textbf{LightCurve or subclass} :

\item[{Raises }] \leavevmode
\textbf{IOError} :
\begin{quote}

If filename is not supplied in {\color{red}\bfseries{}**}kwargs
\end{quote}

\end{description}\end{quote}

\end{fulllineitems}



\chapter{Indices and tables}
\label{index:indices-and-tables}\label{index:welcome-to-lightcurve-s-documentation}\begin{itemize}
\item {} 
\emph{genindex}

\item {} 
\emph{modindex}

\item {} 
\emph{search}

\end{itemize}


\renewcommand{\indexname}{Python Module Index}
\begin{theindex}
\def\bigletter#1{{\Large\sffamily#1}\nopagebreak\vspace{1mm}}
\bigletter{l}
\item {\texttt{lightcurve.cos}}, \pageref{index:module-lightcurve.cos}
\item {\texttt{lightcurve.io}}, \pageref{index:module-lightcurve.io}
\item {\texttt{lightcurve.lightcurve}}, \pageref{index:module-lightcurve.lightcurve}
\end{theindex}

\renewcommand{\indexname}{Index}
\printindex
\end{document}
